% TEMPLATE for Usenix papers, specifically to meet requirements of
%  USENIX '05
% originally a template for producing IEEE-format articles using LaTeX.
%   written by Matthew Ward, CS Department, Worcester Polytechnic Institute.
% adapted by David Beazley for his excellent SWIG paper in Proceedings,
%   Tcl 96
% turned into a smartass generic template by De Clarke, with thanks to
%   both the above pioneers
% use at your own risk.  Complaints to /dev/null.
% make it two column with no page numbering, default is 10 point

% Munged by Fred Douglis <douglis@research.att.com> 10/97 to separate
% the .sty file from the LaTeX source template, so that people can
% more easily include the .sty file into an existing document.  Also
% changed to more closely follow the style guidelines as represented
% by the Word sample file. 

% Note that since 2010, USENIX does not require endnotes. If you want
% foot of page notes, don't include the endnotes package in the 
% usepackage command, below.

% F6, F11, F6, F6, F7 to compile.

% This version uses the latex2e styles, not the very ancient 2.09 stuff.
\documentclass[letterpaper,twocolumn,10pt]{article}
\usepackage{usenix,epsfig,endnotes}
\begin{document}

%don't want date printed
\date{}

%make title bold and 14 pt font (Latex default is non-bold, 16 pt)
\title{\Large \bf EMSS: Entity Matching in a Semi-Supervised way}




%for single author (just remove % characters)
\author{
{\rm Jia R. Wu}\\
University of Waterloo
\and
{\rm Shaokai Wang}\\
University of Waterloo
% copy the following lines to add more authors
% \and
% {\rm Name}\\
%Name Institution
} % end author

\maketitle

% Use the following at camera-ready time to suppress page numbers.
% Comment it out when you first submit the paper for review.
\thispagestyle{empty}


\subsection*{Abstract}



\section{Introduction}
Entity matching (EM), also known as Entity Resolution (ER), in the world of data management refers to resolving duplicate entities to a single entity. Magellan is an end-to-end entity matching framework that utilizes machine learning to perform entity matching. Active learning in the context of machine learning describes the process of a user actively providing labels for model training. This paradigm of active human labeling is referred to as human-in-the-loop machine learning. In this following paper we are interested in applying active learning techniques to machine learning for Entity Matching. 


We provide a comparison of Entity Matching across popular systems such as Magellan, Dedupe.io and JedAI. Comparisons are conducted with the standard DBLP-ACM dataset. Additionally, we provide a fully reproducible and containerized environment for execution requiring no external dependencies other than Docker (reference here). Finally, we extend Magellan with semi-supervised learning and demonstrate that comparable accuracy for EM can be achieved with less samples. 
\\

We make the following contributions:
\begin{itemize}
  \item An evaluation of EM across common systems
  \item A framework for reproducible execution
  \item An active-learning extension for EM (semi-supervised)
\end{itemize}



\section{Related Work}
Perhaps don't need subsections here. A list may be sufficient.

Talk about dedupe primarily since it is what is being utilized

There has been much work done throughout the years to address entity matching. Efforts ranging from traditional machine learning solutions such as 

Talk about structured and unstructured EM


\subsection{Magellan}
Magellan is described as an end-to-end EM framework as it provides the user with a toolset to conduct EM from the beginning of the workflow to the end. 

tools to support the user through various EM scenarios, provides user guides 


\subsection{Dedupe}
Discuss why they have no publication


\subsection{JedAI}
Talk about what this is.


\section{Methods}
Have a section on why the DBLP-ACM benchmark was used.
Have a section on why we used Docker.
Describe what was done to extend. (Active learning part)


\section{Results}
Describe the performance of using ActiveLearning in Magellan.


\noindent\fbox{
    \parbox{\columnwidth}{
    	\centering
        \textit{Add Key Magellan Result/Conclusion Here}
    }
}
\\\\

\section{Challenges}
Describe the ease of use of each system here.
Describe the potential performance of each system here.

Initially the challenge was determining why Magellan had not attempted to build 








\section{Threats To Validity}


\section{Future Work}


\section{Acknowledgments}


\section{Availability}\label{Availability}
All relevant scripts and data can be retireved from the following repository:
\begin{center}
{\tt https://github.com/JRWu/cs848w20}
\end{center}

{\footnotesize \bibliographystyle{acm}

\theendnotes

\newpage
\bibliography{bibliography}}

\end{document}

