% TEMPLATE for Usenix papers, specifically to meet requirements of
%  USENIX '05
% originally a template for producing IEEE-format articles using LaTeX.
%   written by Matthew Ward, CS Department, Worcester Polytechnic Institute.
% adapted by David Beazley for his excellent SWIG paper in Proceedings,
%   Tcl 96
% turned into a smartass generic template by De Clarke, with thanks to
%   both the above pioneers
% use at your own risk.  Complaints to /dev/null.
% make it two column with no page numbering, default is 10 point

% Munged by Fred Douglis <douglis@research.att.com> 10/97 to separate
% the .sty file from the LaTeX source template, so that people can
% more easily include the .sty file into an existing document.  Also
% changed to more closely follow the style guidelines as represented
% by the Word sample file. 

% Note that since 2010, USENIX does not require endnotes. If you want
% foot of page notes, don't include the endnotes package in the 
% usepackage command, below.

% F6, F11, F6, F6, F7 to compile.

% This version uses the latex2e styles, not the very ancient 2.09 stuff.
\documentclass[letterpaper,twocolumn,10pt]{article}
\usepackage{usenix,epsfig,endnotes}
\begin{document}

%don't want date printed
\date{}

%make title bold and 14 pt font (Latex default is non-bold, 16 pt)
\title{\Large \bf EMSS: Entity Matching in a Semi-Supervised way}




%for single author (just remove % characters)
\author{
{\rm Jia R. Wu}\\
University of Waterloo
\and
{\rm Shaokai Wang}\\
University of Waterloo
% copy the following lines to add more authors
% \and
% {\rm Name}\\
%Name Institution
} % end author

\maketitle

% Use the following at camera-ready time to suppress page numbers.
% Comment it out when you first submit the paper for review.
\thispagestyle{empty}


\subsection*{Abstract}


\section{Introduction}


We make the following contributions:
\begin{itemize}
  \item A comparison of EM 
  \item A framework for evaluation
\end{itemize}

\section{Methods}

\subsection{Environment}

\subsection{Docker}

\subsection{Magellan}

\subsection{Dedupe}



\section{Results}
\subsection{Magellan}


\noindent\fbox{
    \parbox{\columnwidth}{
    	\centering
        \textit{Add Some Result Here}
    }
}
\\\\



\subsection{Dedupe}
\\

\noindent\fbox{
    \parbox{\columnwidth}{
    	\centering
        \textit{Add Dedupe Result Here}
    }
}
\\\\




\section{Threats To Validity}


\section{Future Work}


\section{Acknowledgments}


\section{Availability}\label{Availability}
All relevant scripts and data can be retireved from the following repository:
\begin{center}
{\tt https://github.com/JRWu/cs848w20}
\end{center}

{\footnotesize \bibliographystyle{acm}

\theendnotes

\newpage
\bibliography{bibliography}}

\end{document}

